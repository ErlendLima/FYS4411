\section{Method}\label{sec:Method}
\subsection{Outline of Program}
The code is divided into two: % twain, heh
a C++ backend and Python frontend. The Python code is a simple wrapper around
the C++ and functions as a high level interface which is used in the Jupyter
notebooks. The C++ is object oriented with \textsf{System} running the
simulation, \textsf{Sampler} extracting interesting observables from the
simulation, and \textsf{Hamiltonian}, \textsf{WaveFunction} and
\textsf{InitialState} and their subclasses are composable objects for
constructing different systems, such as spherical trap with non-interacting bosons. 

The Jupyter notebooks are available in the GitHub repository, and run reproduce
all presented herein.

\subsection{Statistical Treatment}
\label{sec:blocking}
When dealing with estimation based on data, either produced experimentally or
numerically, a proper statistical treatment is appropriate.

Perhaps the most common estimator in statistics is the mean:

\begin{equation}
	\label{eq:mean}
	\bar{X} = \frac{1}{N}\sum_{i=1}^{N}{X_i},
\end{equation}
where $X_i$ are identically distributed random variables, but not necessarily
independent. To establish the uncertainty of the estimator, it necessary to
calculate $V(\bar{X})$. By using the linearity of the variance, we have
\newcommand{\Cov}{\mathrm{Cov}}
\begin{align*}
\label{eq:estimator var}
  V(\bar{X}) &= V(\frac{1}{N}\sum_{i=1}^{N}{X_i})\\
             &= \frac{1}{N^2}V(\sum_{i=1}^{N}{X_i})\\
             &=\frac{1}{N^2}\sum_{i,j}{\Cov(X_i, X_j)}\numberthis
\end{align*}

where $\Cov(X_i, X_j)$ is the covariance of $X_i$, and $X_j$. Under the
assumption that the variables are independent, which is often approximately
true, $\Cov(X_i, X_j) = 0$, for $i \neq j$. The variance of the mean is thus simplified to 
\begin{align*}
  V(\bar{X}) &= \frac{1}{N^2}\sum_{i}{Cov(X_i, X_i)} \\
  &= \frac{1}{N^2}\sum_{i}{Var(X_i)} \\
	&=\frac{N}{N^2}Var(X) = \frac{Var(X)}{N}
\end{align*}
where it has been used that the variables are identically distributed.
This is a convenient result, as one can calculate the variance of estimator
based on the variance of random variables $X$ that enter it. This is typically done using the sample variance
\begin{equation*}
	V(X) \approx \frac{1}{N}\sum_{i=1}^{N}{(x_i - \bar{x})^2},
\end{equation*}
where $x_i$ is some sample and $\bar{x} = \frac{1}{N}\sum{x_i}$ is the sample mean.
Problems arise when the data is correlated, as the neglect of cross terms
present in \autoref{eq:estimator var} causes a underestimation of the variance
of the estimator. This is very important to bear in mind when working with VMC,
as data produced by simulating many particles tends to be highly correlated. 


Made popular by (kilde), blocking is a renormalization method for treating correlated data when calculating the variance of $\bar{X}$. Blocking is defined by a recursive downsampling of data. Starting with a vector containing the original data $\vec{X}_0$, blocking of degree $d$ is obtained by performing

\begin{equation*}
	\left(\vec{X}_{i+1}\right)_k = \frac{1}{2}\left(\left(\vec{X}_{i}\right)_{2k -1} + \left(\vec{X}_{i}\right)_{2k}\right),
\end{equation*}
for all $1 \leq i \leq d-1$. 
\newline
It can be shown that the quantity 

\begin{equation*}
	V(\bar{X_k}) \approx \frac{\hat{\sigma}_k^2}{n_k} +  e_k
\end{equation*}

is invariant with respect to the degree of blocking $k$, where $\hat{\sigma}_k^2$ is the sample variance of $\vec{X}_k$, and $e_k$ is know as the truncation error. Under the assumption that the original data $\vec{X}_0$ is asymptotically uncorrelated, (kilde) showed that $e_k$ is decreasing, and can be make small given big enough $k$. Therefore, for sufficiently large $k$

\begin{equation}\label{eq:trunc}
V(\bar{X}_0) = V(\bar{X}_k) \approx \frac{\hat{\sigma}_k^2}{n_k} +  e_k \approx \frac{\hat{\sigma}_k^2}{n_k}
\end{equation}

In this way, the sample variance of the sufficiently blocked data does not suffer from underestimation, as the correlation is first eliminated. 
  

\subsection{Gradient Descent}
Variational Monte Carlo involves varying the variational parameter \(\alpha\)
to find its optimal value. This is accomplished using gradient descent. The
iterative method updates its guess at \(\alpha\) by steeping in the direction of
the largest change in the local energy with respect to \(\alpha\)

\begin{align*}
  \alpha_{n+1} &= \alpha_{n} - \eta\pdv{E_{L}}{\alpha}\left( \alpha_{n} \right)
\end{align*}

As optimizing \(\alpha\) is comparatively simple, no more fancy a method is
needed.
%%% Local Variables:
%%% mode: latex
%%% TeX-master: "../main"
%%% End:
