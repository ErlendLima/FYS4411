\section{Theory}\label{sec:Theory}
\subsection{Variational Monte Carlo}
In order to find a good candidate wavefunction for a given potential, one can
employ the \textit{variational principle}. One starts by guessing a trial
wavefunction \(\ket{\Psi_{T}}\) and estimating the trial energy, which is
guaranteed to be equal to or higher than the true ground state energy \(E_{0}\):
\begin{equation}
  \label{eq:1}
  E_{0} \le E = \frac{\ev{H}{\Psi_{T}}}{\ip{\Psi_{T}}}
\end{equation}

If \(\ket{\Psi_{T}}\) is an eigenfunction of the Hamiltonian, the variance
\(\sigma^{2}\) will be minimal

\begin{equation*}
  \sigma^{2} = \frac{\ev{H^{2}}{\Psi_{T}}}{\ip{\Psi_{T}}} -\left( \frac{\ev{H}{\Psi_{T}}}{\ip{\Psi_{T}}} \right)^{2} = 0
\end{equation*}

The variational principle expands on this idea by letting \(\ket{\Psi_{T}}\) be
a functional class of a \textit{variational parameter} \(\alpha\). By varying
\(\alpha\) one can find the optimal trial wavefunction within the functional
class by minimizing \(\sigma^{2}\). 

Only a small collection of potentials have analytical solution using the
variational principle. For most potentials, one must numerically
integrate~\eqref{eq:1} using Monte Carlo integration.

For a stochastic variable \(x\) with probability density function \(p(x)\), the
average \(\left< x \right>\) is defined as
\begin{equation*}
  \left< x \right> = \int_{\mathbb{R}} xp(x)\dd x
\end{equation*}
By sampling the stochastic variable \(M\) times, the average can be approximated
by 

\begin{equation*}
  \expval{x} = \int_{\mathbb{R}} xp(x)\dd x \approx \frac{1}{M}\sum_{i=1}^{M}x_{i}p(x_{i})
\end{equation*}

Applying this to an observable \(\mathcal{O}\), we have

\begin{align*}
  \expval{\mathcal{O}} &= \ev{\mathcal{O}}{\Psi}\\
                       &= \int \dd \vb{r} \Psi^{*}\mathcal{O}\Psi \\
                       &= \int \dd \vb{r} \abs{\Psi}^{2} \frac{1}{\Psi}\mathcal{O}\Psi\\
  &= \frac{1}{M} \sum_{i=1}^{M}p(\vb{r})\mathcal{O}_{L}
\end{align*}

where \(\abs{\Psi}^{2}\) is defined as the probability density function, and
\(\frac{1}{\Psi}\mathcal{O}\Psi\) the \textit{local operator}.

The local trial energy can then be defined as
\begin{equation*}
  E_{L} =\frac{1}{\Psi_{T}}H\Psi_{T}
\end{equation*}
which can be computed using Monte Carlo integration as

\begin{align*}
  \expval{E_{L}} \approx \frac{1}{M}\sum_{i=1}^{M} p(\vb{r}_{i})E_{L}(\vb{r}_{i})
\end{align*}

The goal is therefore to minimize minimizing \(\sigma^{2} = \expval{E_{L}^{2}} -
\expval{E_{L}}^{2}\) over the variational parameter \(\alpha\).

\subsection{Gradient Descent}
\subsection{Potentials}
\subsection{Drift Force}
\subsection{Onebody Density}
