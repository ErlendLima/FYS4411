\section{Theory}\label{sec:Theory}
\subsection{Variational Monte Carlo}
In order to find a good candidate wavefunction for a given potential, one can
employ the \textit{variational principle}. One starts by guessing a trial
wavefunction \(\ket{\Psi_{T}}\) and estimating the trial energy, which is
guaranteed to be equal to or higher than the true ground state energy \(E_{0}\):
\begin{equation}
  \label{eq:1}
  E_{0} \le E = \frac{\ev{H}{\Psi_{T}}}{\ip{\Psi_{T}}}
\end{equation}

If \(\ket{\Psi_{T}}\) is an eigenfunction of the Hamiltonian, the variance
\(\sigma^{2}\) will be minimal

\begin{equation*}
  \sigma^{2} = \frac{\ev{H^{2}}{\Psi_{T}}}{\ip{\Psi_{T}}} -\left( \frac{\ev{H}{\Psi_{T}}}{\ip{\Psi_{T}}} \right)^{2} = 0
\end{equation*}

The variational principle expands on this idea by letting \(\ket{\Psi_{T}}\) be
a functional class of a \textit{variational parameter} \(\alpha\). By varying
\(\alpha\) one can find the optimal trial wavefunction within the functional
class by minimizing \(\sigma^{2}\). 

Only a small collection of potentials have analytical solution using the
variational principle. For most potentials, one must numerically
integrate~\eqref{eq:1} using Monte Carlo integration.

For a stochastic variable \(x\) with probability density function \(p(x)\), the
average \(\left< x \right>\) is defined as
\begin{equation*}
  \left< x \right> = \int_{\mathbb{R}} xp(x)\dd x
\end{equation*}
By sampling the stochastic variable \(M\) times, the average can be approximated
by 

\begin{equation*}
  \expval{x} = \int_{\mathbb{R}} xp(x)\dd x \approx \frac{1}{M}\sum_{i=1}^{M}x_{i}p(x_{i})
\end{equation*}

Applying this to an observable \(\mathcal{O}\), we have

\begin{align*}
  \expval{\mathcal{O}} &= \ev{\mathcal{O}}{\Psi}\\
                       &= \int \dd \vb{r} \Psi^{*}\mathcal{O}\Psi \\
                       &= \int \dd \vb{r} \abs{\Psi}^{2} \frac{1}{\Psi}\mathcal{O}\Psi\\
  &= \frac{1}{M} \sum_{i=1}^{M}p(\vb{r})\mathcal{O}_{L}
\end{align*}

where \(\abs{\Psi}^{2}\) is defined as the probability density function, and
\(\frac{1}{\Psi}\mathcal{O}\Psi\) the \textit{local operator}.

The local trial energy can then be defined as
\begin{equation*}
  E_{L} =\frac{1}{\Psi_{T}}H\Psi_{T}
\end{equation*}
which can be computed using Monte Carlo integration as

\begin{align*}
  \expval{E_{L}} \approx \frac{1}{M}\sum_{i=1}^{M} p(\vb{r}_{i})E_{L}(\vb{r}_{i})
\end{align*}

The goal is therefore to minimize minimizing \(\sigma^{2} = \expval{E_{L}^{2}} -
\expval{E_{L}}^{2}\) over the variational parameter \(\alpha\).

\subsection{Gradient Descent}
The optimal value for the variational parameter is found by gradient descent.

\subsection{The System}
\subsubsection{The Potentials}

The Hamiltonian under investigation describes \(N\) bosons in a potential trap,
and is on the form

\begin{equation*}
  H = \sum_{i=1}^{N}\left( \frac{-\hbar^{2}}{2m}\nabla^{2}_{i} + V_{\text{ext}}(\vb{r}_{i})\right) + \sum_{i<j}^{N}V_{\text{int}}(\vb{r}_{i}, \vb{r}_{j})
\end{equation*}

where \(V_{\text{ext}}\) is the external potential of the trap while
\(V_{\text{int}}\) is the internal potential between the particles. 
The external potential has an elliptical form, being anisotropic in the \(z\)-direction:

\begin{equation}
  \label{eq:2}
  V_{\text{ext}}(\vb{r}) = \frac{1}{2}m\left( \omega\left[ x^{2} + y^{2} \right] + \omega_{z}z^{2} \right)
\end{equation}

The internal potential is a hard shell potential, being infinite for distances
where two bosons overlap:

\begin{equation*}
  V_{\text{int}} = \begin{cases}
    \infty, &\text{for } |r_{i} - r_{j}| \le 0\\
    0, &\text{otherwise}
    \end{cases}
\end{equation*}

\subsubsection{The Trial Wavefunction}

The elliptical spherical trap~\eqref{eq:2} represents a harmonic oscillator. As
the trial wavefunction should be as close as possible to the expected true
wavefunction, a reasonable guess at its shape is the eigenfunction of
harmonic oscillators, namely Gaussian functions. For a \(N\)-bosonic system the
trial wavefunction is therefore

\begin{align*}
  h(\vb{r}_{1}, \ldots, \vb{r}_{N}, \alpha, \beta) &= \prod_{i=1}^{N}g(\vb{r}_{i}, \alpha, \beta)\\
  &= \exp{-\sum_{i=1}^{N}\left( x_{i}^{2} + y_{i}^{2}+\beta z_{i}^{2} \right)}
\end{align*}
with \(g\) the onebody function.
The internal potential should cause the wavefunction to decrease continuously
down to zero as the distance of two particles goes to zero. Once such possible
function is

\begin{equation*}
  f(a, \vb{r}_{i}, \vb{r}_{j}) = \begin{cases}
    0, &|\vb{r}_{i} - \vb{r}_{j}| \le a\\
    1 - \frac{a}{\abs{r_{i}-r_{j}}}, &\text{otherwise}
    \end{cases}
\end{equation*}

Combining both potential contributions, the complete trial wave function is
therefore

\begin{equation*}
  \Psi_{T}(\vb{r}, \alpha, \beta, a) = \exp{-\alpha\sum_{i=1}^{N}\left( x_{i}^{2} + y^{2}_{i} + \beta z^{2}_{i} \right)}
  \prod_{i<j}^{N}f(a, \vb{r}_{i}, \vb{r}_{j})
\end{equation*}

\subsubsection{Non-interacting Case}
For non-interacting bosons in a spherical with \(\beta = 1\) and \(a = 0\) the
system reduces to spherical harmonic oscillators where analytical solutions are
available. The trial wavefunction reduces to simply the product of one body
elements

\newcommand{\psit}{\Psi_T(\vb{r})}
\newcommand{\onebody}{\prod_{i}^{N}\exp{-\alpha\left[\left( x_i^2 + y_i^2 + \beta
      z_i^2\right)\right]}}
\begin{align*}
  \psit = \prod_{i}^{N}\exp[-\alpha\left( x_{i}^{2}+y_{i}^{2}+z_{i}^{2} \right)] = \prod_{i}^{N}\exp(-\alpha \abs{r_{i}}^{2})
\end{align*}

while the Hamiltonian reduces to

\begin{align*}
  H = \sum_{i}^{N} \frac{-\hbar^{2}}{2m}\nabla_{i}^{2} + \frac{1}{2}m\omega^{2}r_{i}^{2}
\end{align*}

which in natural units is

\newcommand{\lapl}[1]{\nabla_{#1}^2}
\begin{align*}
  H = \frac{1}{2}\sum_{i}^{N} -\frac{1}{m}\lapl{i} + m\omega^{2}r_{i}^{2}
\end{align*}

Applying the Hamiltonian gives the local energy as

\begin{align*}
  E_{L} = \frac{\alpha d}{m} N + \left( \frac{1}{2}m\omega^{2} - \frac{2\alpha^{2}}{m} \right)\sum_{i}^{N}r_{i}^{2}
\end{align*}
 where \(d\) is the dimension. As the factor \(\sum r_{i}^{2}\) is always
 positive, its term should be minimized, which is accomplished by setting
 \(\alpha = \frac{m\omega}{2}\), giving a minimal local energy of

 \begin{align*}
   E_{L} = \frac{\omega d N}{2}
   \end{align*}

\subsubsection{Interacting Case}
The local energy for the full interacting case is much more complicated. The
full computation is deferred to appendix~\ref{appendix:le}.





\subsection{Drift Force}
\subsection{Onebody Density}
