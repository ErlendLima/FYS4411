\begin{minipage}{\columnwidth}
\section{Conclusion}\label{sec:Conclusion}
Solving many-body systems numerically is a daunting task, but aided by
Metropolis-Hastings it becomes feasible for small systems. It was shown that it
is an accurate algorithm for sampling states and discovering the expected local
energy for different trial wave functions.

As local energies produced in this fashion tends to be highly correlated,
blocking proved invaluable for removing correlation and establishing faithful
estimates of statistical error introduced by Monte-Carlo. The severe
underestimation of variance when not using blocking was shown in~\cref{fig:blocking1}.

Importance sampling was shown in~\cref{fig:blocking1} to produce data less
correlated than that produced by brute force, in turn yielding more effective
data, as a lower degree of blocking was required to make the data uncorrelated.  

The ground state energy in the space of our trial wave function was calculated
to a precision of $0.005\%$, given that the optimal $\alpha$ was
correct(\cref{tab:energies}). The precision was lower for $50$ and $100$
bosons, $0.02\%$ and $0.1\%$ repectivly. As previously stated, a higher number
of particles results in more strongly correlated local energies, and therefore
less effective data.  

The radial onebody density reveal an important characteristic of the system,
namely that the hardshell potential acts repulsively, causing the distribution
of particles to be wider in the interacting case than to the non-interacting.
For a higher number of particles for a given confining potential, the widening
was more pronounced, which goes to proving the same point.   

\end{minipage}

\section{Future Work}
Some manual labor is involved in picking the optimal degree of blocking and step
length. These can be found automatically by seeing where the variance stabilizes
and the acceptance rate, respectively. As the time spent manually performing
these tasks was less than the time it would take to implement these solutions,
this was not done, but remain low-hanging fruits for future work.

The project can be easily expanded to treat fermionic systems. As we have
developed a general solver class for VMC problems, only 
the trial wave function must be replaced by a appropriate anti-symmetric
function that fits the problem we want to investigate, such as electrons in
a harmonic potential.    
%%% Local Variables:
%%% mode: latex
%%% TeX-master: "../main"
%%% End:
