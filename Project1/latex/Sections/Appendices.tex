\appendix
\section{Local Energy}\label{appendix:le}
\subsection{The Gradient and Laplacian}

The full trial wavefunction can be written as

\begin{align*}
  \psit = \prod_{i}^{N}\phi(\vb{r}_{i}) \cdot \exp\left( \sum_{j<k}u(r_{jk}) \right)
\end{align*}
where \(u = \ln f(r_{ij})\) and \(r_{ij} = \abs{\vb{r}_{i} - \vb{r}_{j}}\).

By the chain rule we can focus on each factor separately when taking the
gradient. The gradient of the first factor is simply the gradient of the one
body element in question:

\begin{align*}
  \nabla_{k} \prod_{i}\phi_{i} = \prod_{i\neq j} \phi_{i} \cdot \nabla_{k}\phi_{k}
\end{align*}

The second factor is at face value slightly more arduous, but it is trivial. The
summation is just a notational trick to ensure that each interaction between all
particles are enumerated once and only once. As the gradient only affects
particle \(k\), only the interactions involving it will be non-zero. Hence the
gradient is

\begin{align*}
  \nabla_{k} \exp\left( \sum_{j<m}u(r_{jm}) \right)
  &= \exp\left( \sum_{j<m}u(r_{jm}) \right)\sum_{l\neq k} \nabla_{k} u(r_{lk})
\end{align*}

The total gradient of the trial wavefunction is therefore
\newcommand{\tikzcircle}[2][red,fill=red]{\tikz[baseline=-0.5ex]\draw[#1,radius=#2] (0,0) circle ;}%
\begin{align*}
  \nabla_{k}\psit &=\left[\underbrace{\prod_{i\neq k} \phi_{i} \cdot \nabla_{k}\phi_{k}}_{\tikzcircle{2pt}} + 
  \underbrace{\prod_{i}\phi_{i}\sum_{l\neq k} \nabla_{k} u(r_{lk})}_{\tikzcircle[green, fill=green]{2pt}}\right]\\
  &\times\exp\left( \sum_{j<m}u(r_{jm}) \right)\\
  &= \left[ \frac{\nabla_{k}\phi_{k}}{\phi_{k}} + \sum_{l\neq k}\nabla_{k}u(r_{lk})\right]\psit
  \numberthis\label{eq:ap3}
\end{align*}

The Laplacian is likewise computed termwise
\begin{align*}
  \nabla_{k}^{2} \tikzcircle{2pt} &= \left[\nabla_{k}^{2}\phi\prod_{i\neq k}\phi_{i} + \nabla_{k}\phi\sum_{i\neq k}\nabla_{k}u\left( r_{ki} \right)\right]\exp\left( \sum_{j<m}u(r_{jm}) \right)
\end{align*}
and

\begin{align*}
  \nabla_{k}^{2}\tikzcircle[green, fill=green]{2pt}
  &=\bigg[ \nabla_{k}\phi_{k}\prod_{i\neq k}\phi_{i} \sum_{l\neq l}\nabla_{k}u(r_{lk})\\
  &+ \prod_{i}\phi_{i}\left( \sum_{l\neq k}\nabla_{k}u\left(r_{lk}\right) \right)^{2}\\
  &+ \prod_{i}\phi_{i}\sum_{l\neq k}\nabla_{k}^{2}u\left( r_{lk} \right)\bigg] \exp\left( \sum_{j<m}u(r_{jm}) \right)
\end{align*}

Combining the results and diving through by \(\Psi_{T}\) then gives

\begin{align*}
  \frac{\nabla_{k}^{2}(\tikzcircle{2pt} + \tikzcircle[green, fill=green]{2pt})}{\Psi_{T}}
  &= \frac{\nabla_{k}^{2}\phi_{k}}{\phi_{k}} + \frac{\nabla_{k}\phi_{k}}{\phi_{k}}\sum_{l\neq k}\nabla_{k}u(r_{lk})
    + \frac{\nabla_{k}\phi_{k}}{\phi_{k}}\sum_{l\neq k}\nabla_{k}u\left( r_{lk} \right)\\
  &+ \left[ \sum_{l \neq k}\nabla_{k}u\left( r_{lk} \right) \right]^{2} + \sum_{l\neq k}\nabla^{2}_{k}u\left( r_{lk} \right)
    \numberthis\label{eq:ap1}
\end{align*}

To compute \(\nabla_{k} u\left( r_{lk} \right)\), a change of variable is
performed, with

\begin{align*}
  \nabla_{k} &= \nabla_{k}\left( r_{kl} \right)\pdv{r_{kl}} = \frac{\vb{r}_{k}-\vb{r}_{l}}{r_{kl}}\pdv{r_{kl}}
\end{align*}
and
\begin{align*}
  \nabla_{k}\left( \frac{\vb{r}_{k}-\vb{r}_{l}}{r_{kl}} \right)
  &= \frac{r_{kl}\nabla_{k}\left( \vb{r}_{k} - \vb{r}_{l}\right) -
  \left( \vb{r}_{k} - \vb{r}_{l} \right)\nabla_{k}r_{kl}}{r_{kl}^{2}}\\
  &= \frac{dr_{kl}^{2} - \left(\vb{r}_{k}  - \vb{r}_{l}\right)\left( \vb{r}_{l}-\vb{r}_{k} \right)}{r_{kl}^{3}}\\
  &= \frac{dr_{kl}^{2} - \vb{r}_{k}^{2} - \vb{r}_{l}^{2} + 2\vb{r}_{k}\vb{r}_{l}}{r_{kl}^{3}}\\
  &= \frac{d-1}{r_{kl}}
\end{align*}

Applying these relations on~\eqref{eq:ap1} and multiplying out the squared sum,
we obtain the desired result

\begin{align*}
  \frac{\nabla_{k}^{2}\psit}{\psit}
  &= \frac{\nabla_{k}^{2}\phi_{k}}{\phi_{k}} + 2 \frac{\nabla_{k}\phi_{k}}{\phi_{k}}\left( \sum_{l\neq k} \frac{\vb{r}_{k}-\vb{r}_{l}}{r_{kl}}u^{\prime}\left( r_{kl} \right) \right)\\
  &+ \sum_{i\neq k}\sum_{j\neq k}\frac{\left( \vb{r}_{k}-\vb{r}_{i} \right)\left( \vb{r}_{k} -\vb{r}_{j} \right)}{r_{ki}r_{kj}}u^{\prime}\left( r_{ki} \right)u^{\prime}\left( r_{kj} \right)\\
  &+ \sum_{l\neq k} \left( u^{\prime\prime}\left( r_{kl} \right)  + \frac{2}{r_{kl}}u^{\prime}(r_{kl})\right)
    \numberthis\label{eq:ap2}
\end{align*}

The full expression involves computing each gradient and derivative. These are
found to be

\begin{align*}
  \nabla_{k}\phi_{k} &= -2\alpha\left( x_{k}\uvec{i} + y_{k}\uvec{j} + \beta z_{k}\uvec{k} \right)\phi_{k}\\
  \nabla_{k}^{2}\phi_{k} &= \left[ 4\alpha^{2}\left(  x_{k}^{2}\uvec{i} + y_{k}^{2}\uvec{j} + \beta^{2} z_{k}^{2}\uvec{k}  \right) - 2\alpha(d - 1 + \beta) \right]\phi_{k}\\
  u^{\prime}(r_{ij}) &= \frac{a}{r_{ij}\left(r_{ij} - a\right)}\\
  u^{\prime\prime}(r_{ij}) &= \frac{a^{2} - 2ar_{ij}}{r_{ij}^{2}\left(r_{ij} - a\right)^{2}}\\
\end{align*}
\subsection{The Local Energy}

The local energy for the full dimensional interactive case is found by
using~\eqref{eq:ap2} and gradients in
\begin{align*}
  E_{L} = \frac{1}{\psit}H\psit
\end{align*}
giving
\begin{align*}
  E_{L} &= -\frac{1}{2m}\sum_{i}\bigg[4\alpha^{2}\left(  x_{k}^{2}\uvec{i} + y_{k}^{2}\uvec{j} + \beta^{2} z_{k}^{2}\uvec{k}  \right)\\
          &-2\alpha(d-1+\beta)\\
        &- 4\alpha\left(  x_{k}\uvec{i} + y_{k}\uvec{j} + \beta z_{k}\uvec{k}  \right)
          \sum_{l\neq k} \frac{\vb{r}_{k}-\vb{r}_{l}}{r_{kl}}u^{\prime}\left( r_{kl} \right)\\
        &+ \sum_{i\neq k}\sum_{j\neq k}\frac{\left( \vb{r}_{k}-\vb{r}_{i} \right)\left( \vb{r}_{k} -\vb{r}_{j} \right)}{r_{ki}r_{kj}}u^{\prime}\left( r_{ki} \right)u^{\prime}\left( r_{kj} \right)\\
        &+ \sum_{l\neq k} \left( u^{\prime\prime}\left( r_{kl} \right)  + \frac{2}{r_{kl}}u^{\prime}(r_{kl})\right)\bigg]\\
        &+ \sum_{i}V_{\text{ext}}(\vb{r}_{i}) + \sum_{i<j}V_{\text{int}}\left( \vb{r}_{i}, \vb{r}_{i} \right)
\end{align*}
\subsection{The Drift Force}
\label{app:drift}
The drift force can likewise be found by inserting the expression for the
gradient~\eqref{eq:ap3} into

\begin{align*}
  \vb{F}_{k} = 2\frac{1}{\psit}\nabla_{k}\psit
\end{align*}
giving
\begin{align*}
  \vb{F}_{k} &= 2 \frac{\nabla_{k}\phi_{k}}{\phi_{k}} + 2\sum_{i\neq k}\frac{\vb{r}_{k}-\vb{r}_{i}}{r_{ki}}u^{\prime}\left( r_{kj} \right)\\
             &= -4\alpha\left(   x_{k}\uvec{i} + y_{k}\uvec{j} + \beta z_{k}\uvec{k}   \right) + 2\sum_{i\neq k} \frac{\vb{r}_{k}-\vb{r}_{i}}{r_{ki}}u^{\prime}\left( r_{ki} \right)\\
\end{align*}
In the non-interactive case this reduces to
\begin{align*}
  \vb{F}_{k} &= -4\alpha\left(   x_{k}\uvec{i} + y_{k}\uvec{j} + \beta z_{k}\uvec{k}   \right)
\end{align*}



\section{Dimensionless Hamiltonian}\label{appendix:dimhamiltonian}
To make the Hamiltonian dimensionless, the following change of variables are
\begin{align*}
  \vb{r} \to \frac{\vb{r}}{a_{\text{ho}}} = \sqrt{\frac{m\omega}{\hbar}}\vb{r} \qquad E \to \frac{E}{\hbar\omega}
  \qquad \nabla \to \sqrt{\frac{\hbar}{m\omega}}\nabla
\end{align*}
giving
\begin{align*}
  H \to H &= \sum_{i}\left( -\frac{\hbar^{2}}{2m}\frac{\omega m}{\hbar}\nabla^{2}_{i}
  + \frac{1}{2}m\frac{\hbar}{m\omega}\left[ \omega^{2}\left( x_{i}^{2} + y_{i}^{2}\right) + \omega_{z}^{2}z_{i}^{2} \right] \right)\\
  &\: + \sum_{i<j} V_{int}(\vb{r}_{i}, \vb{r}_{j})\\
  &= \frac{\hbar\omega}{2}\sum_{i}\left(  -\nabla_{i}^{2} + \left[ x_{i}^{2} + y_{i}^{2} + \gamma^{2} z_{i}^{2} \right]\right)
    + \sum_{i<j} V_{int}(\vb{r}_{i}, \vb{r}_{j})\\
\end{align*}
where the relation \(\gamma = \omega_{z}/\omega\) was made. Applying the \(E\)
substitution and noting that \(V_{int}\) does not change if multiplied by any constant, the final result is obtained: 

\begin{align*}
  H &= \frac{1}{2}\sum_{i}\left(  -\nabla_{i}^{2} + \left[ x_{i}^{2} + y_{i}^{2} + \gamma^{2} z_{i}^{2} \right]\right)
    + \sum_{i<j} V_{int}(\vb{r}_{i}, \vb{r}_{j})\\
\end{align*}